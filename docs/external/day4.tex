\documentclass[12pt]{article}

\title{Day \#4: Miscellany, math, and then some more}
\author{Sean Laverty}
\date{Wednesday, August 30, 2023}

\usepackage{geometry}
\usepackage{amssymb}
\usepackage{amsmath}
\usepackage{xcolor}

\begin{document}

\maketitle
\newpage
\section{x}
%%\usepackage{indentfirst}
\noindent\indent As you write, you might need to \emph{emphasize} certain words or phrases.  We can also explicitly \textit{italicize}, \textbf{bold}, \underline{underline}.
%%, or \sout{strikeout}. %% this might use a package

There are also commands within mathematical statements, for example, \(\mathbf{x}\) which is one way of communicating vectors.  Sometimes we talk about sets of numbers, for example, \(\mathbb{R} \text{ or } \mathbb{Z}\).  Other times we need fancy letters in calligraphy, \(\mathcal{A}\). 

Occasionally, we need to change the size of certain words or parts of text.  We might want to make something {\Huge{Huge}}. Or {\Large{Large}}, or {\large{large}}. Maybe {\small{small}}, or {\footnotesize{footnote size}}, or {\scriptsize{script size}}, {\tiny{tiny}}.  I believe we have {\LARGE{LARGE}} and {\normalsize{normal}}.

We need to ``escape'' certain symbols that are otherwise treated like commands.  For example the dollar sign is \$.  To print a slash we can use \slash or \textbackslash, but a related command requires a math mode \(\backslash\).

For a vector we need that little line on top, \(\overline{v}\) or \(\vec{v}\).  For actuarial science, you might want to explore \verb|\usepackage{actuarialsymbol}|.  To highlight this, we used commands like \verb|\Large| above to adjust the size of fonts.  For longer statements we can use \begin{verbatim} 
%% preamble
\title{Day \#4: Miscellany, math, and then some more}
\author{Sean Laverty}
\date{Wednesday, August 30, 2023}

\usepackage{geometry}
\usepackage{amssymb}
\usepackage{amsmath}
\usepackage{xcolor}
\end{verbatim}

Consider the polynomial defined by \(p_n(x)\), which we expand below,
\begin{align}
%% \nonumber before \\ suppresses line numbering
%% \label{eq::text} allows us to track/reference that equation by number
p_{n}(x) &= \sum_{k=0}^n a_{k}x^{k}\label{eq::sumform}\\
& = a_{0}x^{0} + a_{1}x^{1} + \dots + a_{n-1}x^{n-1} + a_{n}x^n\nonumber\\
p_{n}(x) & = a_{0} + a_{1}x + \dots + a_{n-1}x^{n-1} + a_{n}x^n\label{eq::poly}
\end{align}
Above in Eq.~\eqref{eq::sumform} we define \(p_{n}(x)\) as a polynomial which we then expand in Eq.~\eqref{eq::poly}.

This was fun but let's add some \textcolor{red}{colors}!  We can actually mix colors too, \textcolor{red!50!blue}{purplish}.  
{\Huge{\textcolor{black!10!white}{shading}}}.
%% color!amount!(second color)
\end{document}
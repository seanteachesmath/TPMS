\documentclass[12pt]{article}

\usepackage{geometry}

\title{Day \#5 (Part B): tables}
\author{Sean Laverty}
\date{Wednesday, September 6, 2023}

\begin{document}

\maketitle
\newpage

Sometimes we need to present information in tables.

Below we center-justify,

\begin{tabular}{c|c|c} %% centered
\(x\) & \(f(x) = x^{2}\) & \(g(x) = e^{x}\)\\
\hline\hline
\(0\) & \(f(0) = 0^{2} = 0 \) & \(g(0) = e^{0} = 1\)\\
\(1\) & \(f(1) = 1^{2} = 1 \) & \(g(1) = e^{1} = e\)\\\end{tabular}

Below we right-justify,

\begin{tabular}{r|r|r} %% right
\(x\) & \(f(x) = x^{2}\) & \(g(x) = e^{x}\)\\
\hline\hline
\(0\) & \(f(0) = 0^{2} = 0 \) & \(g(0) = e^{0} = 1\)\\
\(1\) & \(f(1) = 1^{2} = 1 \) & \(g(1) = e^{1} = e\)\\\end{tabular}

Below we left-justify,

\begin{tabular}{lll} %% l and | look identical
\(x\) & \(f(x) = x^{2}\) & \(g(x) = e^{x}\)\\
\hline\hline
\(0\) & \(f(0) = 0^{2} = 0 \) & \(g(0) = e^{0} = 1\)\\
\(1\) & \(f(1) = 1^{2} = 1 \) & \(g(1) = e^{1} = e\)\\\end{tabular}

Below we mix all sorts of justification,

\begin{tabular}{cp{2in}r} %% l and | look identical
\(x\) & \(f(x) = x^{2}\) & \(g(x) = e^{x}\)\\
\hline\hline
\(0\) & \(f(0) = 0^{2} = 0 \) & \(g(0) = e^{0} = 1\)\\
\(1\) & \(f(1) = 1^{2} = 1 \) & \(g(1) = e^{1} = e\)\\\end{tabular}

\newpage
Here is some text about the table below. I don't know what I will put in the table yet. Any ideas?

\begin{tabular}{c|cc}
& Yes & No \\
\hline
Pineapple & 3 & 9\\
Anchovies & 1 & everyone else\\
Mushrooms & 7 & 3\\
Balsamic vinegar and fig & 5 & 4\\
\hline
& 16 & who knows
\end{tabular}
\end{document}









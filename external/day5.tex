\documentclass[12pt]{article}

\usepackage{geometry}

\title{Day \#5: lists}
\author{Sean Laverty}
\date{Wednesday, September 6, 2023}

\begin{document}

\maketitle
\newpage

Sometimes it is useful to list information. For example,
\begin{itemize}
\item Computer Science
\item Actuarial science
% describing the math major
\item Mathematics
% a few key classes
\begin{itemize}
\item Calculus
% a few key ideas
\begin{itemize}
\item Limits
\item Derivatives
\item Integrals
\item Applications
\end{itemize}
% on to FOAM
\item FOAM (Foundations of Advanced Mathematics)
\item Differential Equations
\item History of Mathematics
\end{itemize}

\item Statistics
\item Philosophy
\end{itemize}


\newpage
We can also number things. For example,
\begin{enumerate}
\item For \(f(x) = x^{2}\), find \(f'(x)\) using,
\begin{enumerate}
\item the limit definition of the derivative
\item the power rule
\end{enumerate}
\item Consider the expression \(t^{2^{2}}\) or \({t^2}^{2}\).  I think the first one is the best way to do this. Alternatively we could write this as \((t^{2})^{2}\).  Doing this any less carefully results in a ``double superscript'' error.
\end{enumerate}

\newpage
For curiosity we can manually change symbols,
\begin{itemize}
\item[\(\dagger\)] this is a ``dagger'' symbol
\item[\(\ddagger\)] this is a ``double dagger'' symbol
\item[\(\circ\)] this is a `circ' symbol, which looks like one of the itemize options already
\end{itemize}

\newpage

In writing proofs, we often need to organize our ideas very carefully.  For example in proof by cases, as follows
\begin{description}
\item[\(n\) is even.]\quad\newline
some calc\quad ulat\qquad ions\newline
I can add more text here. I can add more text here. I can add more text here.  I can add more text here. I can add more text here.
\item[\(n\) is odd.] some more calculations
\end{description}

How to override spacing.
\begin{description}
\item[with the quad command]\quad\newline text goes below
\item[with the slash-comma trick]\,\newline text goes below
\item[with the phantom command]\phantom{}\newline text goes below
\item[with your own trick] I don't know what this is figure it out yourself.
\end{description}

\newpage
To work on a resume, we could organize with a list.
\begin{description}
\item[Education]\,\newline
\item[Employment History]\,\newline
\item[Volunteer and Service]\,\newline
\item[Awards and Honors]\,\newline
\item[Skills and Certifications] {\LaTeX}, python (beginner), R (intermediate), sage, Microsoft Office
\item[References] Technically \emph{not} part of the resum\'{e}, but provided alongside as a separate document
\end{description}










\end{document}
\documentclass[landscape, 12pt]{article} %% others include: book, beamer (slides)

%% add titlepage
\author{Sean Laverty}
\title{Day \#3: More math!} %% same for \$
\date{Monday, August 28, 2023}

\usepackage{amsmath}
\usepackage[top = 1in, bottom = 1in, left = 1in, right = 3in]{geometry} %% or [margin = 1in]
\usepackage{setspace}

\begin{document}
\doublespacing %% or \onehalfspacing
\maketitle
\newpage

Previously we discussed the integral. We continue that discussion today.  Consider the indefinite integral \(\int x\,dx\).         Consider the indefinite integral \(\displaystyle \int x\,dx\).       Consider the indefinite integral \[\int x\,dx.\]

Suppose we needed to calculate the definite integral of some function.  Consider the example below,
\begin{align} %% alignment at &
\int_{-1}^{1} x^2\,dx &= \left.\left(\frac{x^3}{3}\right)\right|_{-1}^{1}\\
 &= \frac{(1)^3}{3} -  \frac{(-1)^3}{3}\\
 &= \frac{1}{3} -  \frac{-1}{3}\\
 \int_{-1}^{1} x^2\,dx &= \frac{2}{3}
\end{align}

Suppose we needed to calculate the definite integral of some function.  Consider the example below,
\begin{align*} %% alignment at &, * suppresses numbering
\int_{-1}^{1} x^2\,dx &= \left.\left(\frac{x^3}{3}\right)\right|_{-1}^{1}\\
% enhance readability
 &= \frac{(1)^3}{3} -  \frac{(-1)^3}{3}\\
%
 &= \frac{1}{3} -  \frac{-1}{3}\\
%
\int_{-1}^{1} x^2\,dx &= \frac{2}{3}
\end{align*}

To customize appearance of something like \(\frac{-1}{3}\), we have a few options.  First, we could write \(-\frac{1}{3}\).  Alternatively, we could insert ``phantom'' space, \(\frac{-1}{\phantom{-}3}\). One wild idea is, \(\frac{-1}{\hfill3\hfill}\). %% this uses \hfill

Practice: Write and solve a simple polynomial limit problem using aligned equations, or a derivative calculation.
\end{document}